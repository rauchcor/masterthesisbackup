%%%%%%%%%%%%%%%%%%%%%%%%%%%%%%%%%%%%%%%%%%%%%%%%%%%%%%%%%%%%%%%%%%%%%%%%%%%%%
\chapter{Introduction}\label{chap:introduction}
%%%%%%%%%%%%%%%%%%%%%%%%%%%%%%%%%%%%%%%%%%%%%%%%%%%%%%%%%%%%%%%%%%%%%%%%%%%%%
\chapterstart

With the rise of social networks, online services quite recently experienced a revolution and thus have had a significant impact on the way private information gets propagated on the Internet. Developers split up solutions into backend and frontend applications and build their backend solutions as Application Programming Interfaces (API) which can be used to propagate services online. Frontend applications or even third-party applications can then consume published APIs. This approach especially gained popularity with the rise of mobile and Single Page Applications (SPA), which heavily depend on a light-weighted fronted. However using APIs to access the business logic means, that the trust relationship is not just between two parties, it can also include third-party applications. This uncertain trust relationship raises concerns of users and developers regarding security and privacy of personal information. Building trust is a substantial part for architects and developers when designing a system. In order to efface the security and privacy concerns of users, advanced access and identity management system are needed [cf. (\cite{Cirani:OBAS}), (\cite{Tkalec:2015}), (\cite{Rossvoll:2013:RUBIM})].


\section{Problem Statement}

In traditional architectures, a third party receives the user’s credentials, to provide the user with access to information from a protected resource. Prasad (2016) points out that sharing the credentials with a third party might lead to security problems. Third parties might be responsible for fatal security gaps, for example by storing passwords in plain-text. Prasad (2016) further states that the compromise of one third-party will then lead to the compromise of the credentials of the end user. As a result, the safety of the secure resources cannot be granted anymore. Moreover, third-parties will receive the comparatively greater amount of access to the user data than needed. Another concern is pointed out by Sakimura et al. (2014), which is that users often are required to create new accounts for each service they want to use. On the one hand, the process of creating a new account for a service can be annoying for the user because they have the burden of keeping track of multiple accounts and remembering multiple credentials. On the other hand, the complex task of managing this accounts can lead to problems like password fatigue and in the worst case to identity theft or the compromise of services [cf. (\cite{Sakimura:OIDCC}), (\cite{Prasad:MMWPT})].

This common problems with a modern application setup, where the API represents the backend application raise the question of, how to find the best way to secure these applications and which of the existing approaches is best for a particular case. Furthermore, what risks come with the use of a specific approach and what is the state of privacy? 


\section{Research Question}

What are feasible ways to implement authentication and authorization for enterprises that depend on adaptability and focus on modern single page applications? How can businesses find out which identity management system is best suited for their use case? 

\section{Hypothesis}

In modern architectures, a third party often is responsible for the implementation of the process of authentication and authorization of the user. Considering the setup of a company and the use case of the authentication and authorization of users, the requirements towards identity management can be very different. Grassi, Garcia, \& L. (2017) suggest for identity management, to rely on standards and specifications that are widespread and documented. Example for tools and standards include Internet protocols for accessing services like REST, SOAP, and XML and federate identity standards for service authentication such as SAML, OAuth, and OpenID Connect.

The focus of this thesis is on modern distributed architectures and ways to provide users secure access to protected resources and confidential management of the user’s credentials for a specific scenario. The hypothesis of this thesis is that the best-suited identity management system for a given use case can be found by conducting a risk assessment and finding accurate Level of Assurance (LOA) and security controls to reduce common identity threats and mitigate risks while keeping a balance between security and usability.

\section{Method}

In order to prove the described hypothesis the following method is applied:

\begin{itemize}
	\item Background and related work about authentication and authorization methods
	\item Risk assessment of a use case
	\item Development of a best practice example
\end{itemize}

The priority was to make extensive research on the topic to get an overview of the current state of authentication and authorization methods. After the research, a case for an assessment is defined and described in detail. Based on this case a risk assessment is executed, and the appropriate level of assurance is a determined. The level of assurance need for this case allows choosing the appropriate technologies to mitigate risks and the impact of errors. With this set of technologies and example solution for a modern enterprise SPA is created to give a best practice example based on the theoretical gathered knowledge.


\chapterend
