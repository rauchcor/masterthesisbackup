%%%%%%%%%%%%%%%%%%%%%%%%%%%%%%%%%%%%%%%%%%%%%%%%%%%%%%%%%%%%%%%%%%%%%%%%%%%%%
\chapter{Conclusion and Outlook}\label{chap:conclusion}
%%%%%%%%%%%%%%%%%%%%%%%%%%%%%%%%%%%%%%%%%%%%%%%%%%%%%%%%%%%%%%%%%%%%%%%%%%%%%
\chapterstart

Taking on a new project takes a lot of consideration and planning. The planning includes thoughtful consideration of the security and privacy measures which should be established to create trustworthiness. An essential part of trustworthiness is identity management. Identity management includes ensuring confidentiality, integrity, and availability. 

Chapter two of this thesis focuses on a careful examination of identity management, especially under three different angels identity proofing, authentication, and assertion. Moreover, the difference between using token-based and server-based authentication is analyzed. In token-based authentication systems, it is by far more comfortable to handle sessions, while in server-based authentication systems, this can be challenging tasks. Particular for a single-page application which typically uses an Application Programming Interface hosted on a different server session management can lead to overhead in server-based authentication systems. The thesis also explains the benefits of using a federation system for Single Sign-on and introduces two possible federation systems for Single Sign-on. The two systems which are explained are SAML and OpenID Connect. Both of the systems provide the service of Single sign-on, but ultimately OpenID Connect provides a lighter weighted approach for mobile and single page applications.

The third chapter undertakes the task of finding the best available authentication and authorization methods for a defined use case that requires a Single Page Application and a separated Application Programming Interface. A mature part of reaching this goal was the risk assessment of a use case which was described beforehand. The risk assessment examines different technical threats and rates and determines the risk of each threat according to an existing approach described in the 'Guide for Conducting Risk Assessment' by NIST (2012). Furthermore, flowcharts were consulted which should help with the decision of choosing an appropriate Level of Assurance and if federating the solution should be considered. This risk assessment leads to three different assurance level which can be categorized into identity proofing, authentication, and assertion. 

Based on the level of assurance determined by the conducted risk assessment a solution was created which is an example of the identity management system developed for an use case. The solution is described in chapter four and was based on the knowledge gained from the related work part and considered the outcome of the risk assessment. This leads to a solution with a stand-alone identity server providing a token service, a separate Application Programming Interface which can host private information and a Single Page Application. 

In my opinion, the risk assessment was helpful but should be conducted to a greater extent for a real-life application. Furthermore, the risk determination was difficult because the values of the assessment approach were described very vague in the 'Guide for Conducting Risk Assessment' by NIST (2012), there was no point of reference for assigning the right values. Nevertheless, the risk assessment gave some excellent pointers and made decisions more manageable. For example, it was easier to decide rather to use bearer assertions or holder-of-key assertions. The implementation of the identity management system which was based on the risk assessment is a very feasible way to implement a token service from scratch including the security controls from the risk assessment. The IdentityServer4 framework was very helpful for building the token service, but I would not recommend using the framework without extended knowledge of the OAuth 2.0 and OpenID Connect specification. For example, to implementation and validation of JWT's for an OpenID Connect identity management application, one should know which algorithms are appropriate to use. Additionally, if the solution in chapter four would be further developed, another risk assessment should be conducted. The key to successful risk management is documentation, review and improvement.

\chapterend
